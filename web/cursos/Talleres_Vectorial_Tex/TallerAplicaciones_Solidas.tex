\documentclass[usepdftitle=false]{beamer}
\usetheme{Berkeley}



\title[]{Vectorial Virtual -- Taller 2, parte 1: Aplicaciones de integrales de funciones escalares sobre regiones s\'olidas}

\date{Abril 2020}

\begin{document}

\begin{frame}
  \titlepage
\end{frame}

\begin{frame}{Ejercicio 1}

Sea $E$ un cono circular recto con radio de la base $R$ y altura $H$. Asuma que $E$ tiene como eje de simetr\'ia al eje $z$, que la base circular esta apoyada sobre el plano $(x,y)$ y que la punta esta hacia arriba. Encuentre la posici\'on del centro de masa $(\overline{x}, \overline{y}, \overline{z})$ del cono, asumiendo que este tiene densidad constante. (Nota: La respuesta debe ser una funci\'on de $R$ y $H$). 

\end{frame}

\begin{frame}{Ejercicio 2 (Distribuci\'on Gaussiana en dos dimensiones)}

Sea $f(x,y)= \frac{1}{2\pi} e^{-\frac{x^2+y^2}{2}}$.
\begin{enumerate}
\item Calcule $\iint_Df(x,y)dA$ donde $D$ es el disco de radio $A$ centrado en el origen (Nota: La respuesta depende de $A$).
\item Verifique que, cuando $A\rightarrow \infty$ la cantidad calculada en la parte $(a)$ converge a $1$.
\item Demuestre, utilizando lo que verifico en la parte $(2)$ que $f(x,y)$ es la densidad de probabilidad de un vector aleatorio $(X,Y)$.
\item Calcule $\mathbb{E}(X)$, $\mathbb{E}(Y)$, ${\rm Var}(X)$ y ${\rm Cov}(X,Y)$.
 
\end{enumerate}
\end{frame}

\begin{frame}{Ejercicio 3: Momentos de inercia de cilindros}

Calcule la funci\'on de densidad y el momento de inercia, alrededor de su eje de simetr\'ia de los siguientes s\'olidos:

\begin{enumerate}
\item Un cilindro s\'olido de masa $M$, radio $R$ y altura $L$, con densidad constante
\item Un cilindro s\'olido de masa $M$, radio $R$ y altura $L$, con densidad proporcional a $r$ (i.e. a la distacia al eje de simetr\'ia).
\iem Un anillo con masa $M$ radio iterior $\frac{R}{2}$, radio exterior $R$ y altura $L$ con densidad constate. 
\end{enumerate}
Nota: Las respuestas son funciones de algunas de las variables $M,R$ y $L$.
\end{frame}

\begin{frame}{Ejercicio 4: Momentos de inercia de esferas}

Calcule la funci\'on de densidad y el momento de inercia, alrededor de un eje que pasa por el centro de los siguientes s\'olidos:

\begin{enumerate}
\item Una esfera s\'olida de masa $M$ y radio $R$, con densidad constante
\item Un esfera s\'olida de masa $M$ y radio $R$  con densidad proporcional a $\rho$ (i.e. a la distacia al origen).
\iem Un casquete esf\'erico con masa $M$, radio interior $\frac{R}{2}$ y radio exterior $R$ con densidad constate. 
\end{enumerate}

\end{frame}

\begin{frame}{Ejercicio 5: Roller Derby}

Si hacemos una carrera entre los objetos de los ejercicios $3$ y $4$ (todos con masa $M$ y radio $R$) haciendo que desciendan sobre un plano inclinado rodando sin deslizarse, en qu\'e orden llegan a la meta? Argumente matem\'aticamente su respuesta usando las f\'ormulas que calculamos en la magistral.

\end{frame}




\end{document}
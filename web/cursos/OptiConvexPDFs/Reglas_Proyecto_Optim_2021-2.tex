
\documentclass[12pt, a4paper]{article}
\usepackage{hyperref}
\hypersetup{
  colorlinks=true,
  linkcolor=blue,
  urlcolor=cyan,
}
\urlstyle{same}
\usepackage[utf8]{inputenc}
\usepackage{amsmath}
\usepackage{amsfonts}
\usepackage{amssymb}
\usepackage{graphicx}


\newtheorem{theorem}{Teorema.}
\newtheorem{lemma}[theorem]{Lema.}
\newtheorem{corollary}[theorem]{Corolario.}
\newtheorem{definition}[theorem]{Definici\'on:}
\newtheorem{example}[theorem]{Ejemplo:}
\newtheorem{problema}[theorem]{Problema:}
\newtheorem{remark}[theorem]{Observaci\'on:}

\usepackage{graphicx}
\usepackage[spanish]{babel}
%\usetheme{default}

\newcommand{\pp}{\mathbb{P}}
\newcommand{\zz}{\mathbb{Z}}
\newcommand{\rr}{\mathbb{R}}
\newcommand{\qq}{\mathbb{Q}}

\usepackage{tikz, tikz-3dplot}

\definecolor{cof}{RGB}{219,144,71}
\definecolor{pur}{RGB}{186,146,162}
\definecolor{greeo}{RGB}{91,173,69}
\definecolor{greet}{RGB}{52,111,72}


\begin{document}
\title{PROYECTO DEL CURSO OPTIMIZACI\'ON CONVEXA (2021-2)}
\maketitle

\section{Objetivos del proyecto}
El objetivo del proyecto es que implementen software (en julia / python) que resuelva alg\'un problema de optimizaci\'on de su inter\'es. El proyecto deben servir para mejorar el uso pr\'actico de software especializado de optimizaci\'on y/o el manejo computacional de datos.

A continuaci\'on listo algunas direcciones posibles pero pueden, si quieren, elegir algo que no este en esta lista. 


\begin{enumerate}
\item Optimizaci\'on para el manejo de capacidad hospitalaria:
\href{https://www.gurobi.com/resource/covid-19-hospital-capacity-management/}{Charla Ghobadi}\\
Problema: Replicar alguna versi\'on de estos resultados para los hospitales de Bogot\'a usando los datos abiertos del distrito.
\item (Traveling Salesman en Colombia) Encuentre el camino m\'as corto que recorre todas las ciudades de Colombia: 
\href{https://www2.oberlin.edu/math/faculty/bosch/tspart-page.html}{Dibujos}
y  
\href{https://developers.google.com/optimization/routing/tsp}{librerias existentes: OR-Tools}
\item  Implemente alguna versi\'on aproximada de rank minimization con aplicaciones como las de ac\'a:
\href{https://www.youtube.com/watch?v=QMybSG5vOgE}{Charla Udell}
\item{Relative entropy minimization: \href{http://users.cms.caltech.edu/~venkatc/cs_rep_mathprog17.pdf}{Paper Chandrasekaran;} 
\href{https://www.youtube.com/watch?v=vUPcyENTbGA}{Charla Chandrasekaran;}
Muy relacionado con el seminario ECO.
}
\item Gaussian kernel factorization via approximation theory:
\href{https://www.youtube.com/watch?v=83R63nfYgJM}{Charla Parrilo}
\item  Switched linear systems and infinite products of matrices:
\href{https://www.youtube.com/watch?v=V71STrm_YyM}{Charla Parrilo}
\item Machine Learning Interpretable y optimizaci\'on: \href{https://www.gurobi.com/resource/develop-more-accurate-machine-learning-models-with-mip/}{Charla Bertsimas}
Proyecto: Implementar el problema de optimal decision trees \href{https://arxiv.org/pdf/2103.15965.pdf}{Paper Aghaei, Gomez, Vayanos;} \href{https://arxiv.org/abs/2007.12652}{Paper Dyn prog;}\href{https://arxiv.org/abs/1904.12847}{Paper sparse}


\item{Algoritmos para optimizaci\'on semidefinida a gran escala. Implementar un buen solver de estos} \href{https://arxiv.org/pdf/1912.02949.pdf}{Paper Udell, Tropp, etc.}

\end{enumerate}


 


\section{Reglas del proyecto}
\begin{enumerate}
\item{El proyecto es un trabajo en grupo. Cada grupo debe ser de m\'aximo $2$ personas (y la nota ser\'a la misma para todos los integrantes del grupo).}
\item{Es necesario leer por lo menos dos art\'iculos pertinentes en cada proyecto.}
\item{Es necesario entregar un software en cada proyecto. }
\item{{\bf Entregas:} La totalidad del proyecto consiste de tres entregas (a hacer en LaTeX, un solo documento por grupo en cada entrega),
\begin{enumerate}
\item{Entrega 1 (Datos del proyecto): Entregar el t\'itulo del proyecto en que van a trabajar, los integrantes del grupo y un p\'arrafo con una descripci\'on del problema en que van a trabajar.}
\item{Entrega 2 (Plan concreto del proyecto): Debe contener: El (o los) modelo(s) matem\'atico(s) del problema que quieren estudiar, {\it los datos que quieren utilizar} y referencias a al menos dos art\'iculos pertinentes. Idealmente debe incluir algunos ejemplos de prueba en los que pueden resolver el problema efectivamente y los resultados que obtienen.}
\item{Entrega 3 (Examen final): La entrega final consiste de cuatro partes,
\begin{enumerate}
\item{Un documento en LaTex explicando los resultados del proyecto. Debe incluir una descripci\'on del problema, del modelo utilizado y de los resultados computacionales obtenidos (con tablas, figuras, etc.) asi como una secci\'on explicando qu\'e dicen los resultados sobre la pregunta original propuesta. Debe ser de a lo m\'as 10 p\'aginas de longitud, siguiendo el formato de los What is? de la AMS
\url{https://www.ams.org/journals/notices/201405/rnoti-p492.pdf}

}
\item{El c\'odigo del programa utilizado para obtener los resultados (en el/los lenguajes que quieran)}
\item{Una charla de $40$ mins con el objetivo de explicar los resultados del proyecto.}
\item{Asistir a todas las charlas de sus compa\~neros el dia del examen final.}
\end{enumerate}
}
\end{enumerate}

}
\end{enumerate}
\end{document}

\documentclass[12pt, a4paper]{article}
\usepackage{hyperref}
\hypersetup{
  colorlinks=true,
  linkcolor=blue,
  urlcolor=cyan,
}
\urlstyle{same}
\usepackage[utf8]{inputenc}
\usepackage{amsmath}
\usepackage{amsfonts}
\usepackage{amssymb}
\usepackage{graphicx}


\newtheorem{theorem}{Teorema.}
\newtheorem{lemma}[theorem]{Lema.}
\newtheorem{corollary}[theorem]{Corolario.}
\newtheorem{definition}[theorem]{Definici\'on:}
\newtheorem{example}[theorem]{Ejemplo:}
\newtheorem{problema}[theorem]{Problema:}
\newtheorem{remark}[theorem]{Observaci\'on:}

\usepackage{graphicx}
\usepackage[spanish]{babel}
%\usetheme{default}

\newcommand{\pp}{\mathbb{P}}
\newcommand{\zz}{\mathbb{Z}}
\newcommand{\rr}{\mathbb{R}}
\newcommand{\qq}{\mathbb{Q}}

\usepackage{tikz, tikz-3dplot}

\definecolor{cof}{RGB}{219,144,71}
\definecolor{pur}{RGB}{186,146,162}
\definecolor{greeo}{RGB}{91,173,69}
\definecolor{greet}{RGB}{52,111,72}


\begin{document}
\title{INSTALACI\'ON DE JULIA para OPTIMIZACI\'ON}
\date{}
\maketitle

Deben seguir los siguientes pasos:

\begin{enumerate}
\item Es necesario tener un editor de Julia que funcione bien. Les recomiendo Visual Studio Code. El lenguaje de programaci\'on Julia, el editor de c\'odigo Visual Studio Code y la extensi\'on para Julia de VSCode se instalan siguiendo los pasos de ac\'a:\\
\url{https://www.julia-vscode.org/docs/dev/gettingstarted/}

\item Instalar el paquete JuMP de Julia (que quiere decir Julia Mathematical Programming y que permite formular problemas de optimizaci\'on de muchos tipos) siguiendo los pasos descritos ac\'a:\\
\url{https://jump.dev/JuMP.jl/stable/installation/#Installation-Guide}

\item JuMP permite modelar problemas de optimizaci\'on. Para resolverlos hay que instalar alg\'un solver (que JuMP llama cuando ustedes le dan el comando optimize!). Un solver bueno y con una licencia amplia es GLPK (que deja resolver problemas lineales y lineales enteros). CSDP es un solver para programaci\'on semidefinida. Consultar el link en $(2)$ sobre detalles de la instalaci\'on de solvers.


\item Despu\'es de tener alg\'un solver instalado intente resolver su propio problema! Puede iniciar ejecutando el ejemplo de ac\'a:\\
\url{https://jump.dev/JuMP.jl/stable/tutorials/Gettingstarted/getting_started_with_JuMP/#Getting-started-with-JuMP}



\end{enumerate}


\end{document}
